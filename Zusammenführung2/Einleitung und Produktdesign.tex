\documentclass[13pt,titlepage]{article}

\usepackage[ngerman]{babel}
\usepackage[utf8]{inputenc}
\usepackage{color, colortbl}
\usepackage[table,xcdraw]{xcolor}
\usepackage{graphicx}

\usepackage{caption}
\usepackage{float}
\restylefloat{table}
\usepackage{verbatim}
\usepackage{prettyref}
\usepackage[gen]{eurosym}
\usepackage[super]{nth}

\usepackage [backend=biber]{biblatex}
\usepackage[left=3cm,
  right=2cm,
  top=2.5cm,
  bottom=2cm,
  ]{geometry}
\def\code#1{\texttt{#1}}

\bibliography{Literatur.bib}

\title{Nachhaltigkeit in Design und Produktion}
\date{16.12.2020}
\author{Markus Schütze}

\begin{document}

\maketitle
\tableofcontents
\newpage

\section {Einleitung}
	Es gibt viele verschiedene M\"oglichkeiten Produkte umweltfreundlicher zu machen.
Eine Möglichkeit Umweltmaßnahmen des Unternehmens im größeren Umfang zu beeinflussen, ist während des Produktlebenszyklus. 
Dieser legt den Umwelteinfluss eines Produkts, von den benutzten Rohmaterialien und der dazu aufgewendeten Energie, bis hin zur Entsorgung fest. 
Das Gesamtziel ist die Reduzierung des ökologischen Fußabdrucks eines Produktes während dem gesamten Produktzyklus positiv zu beeinflussen. 
Das bringt uns zur Idee der Kreislaufwirtschaft(vgl. Abb. \ref{fig:LinearKreislaufwirtschaft}). 
Diese bietet eine Alternative zur traditionellen Wegwerfgesellschaft oder auch linearen Wirtschaft.
In der Kreislaufwirtschaft ist das Ziel das Produkt so lange wie m\"oglich zu benutzen und bei der Entsorgung so viele eingesetzte Ressourcen wie m\"oglich zur\"uckzuerlangen. 
Um dies zu erreichen sollen Produkte m\"oglichst hochwertig mit hoher Lebensdauer erzeugt werden, damit diese repariert, aufgearbeitet und wiederverwendet werden können\protect\footnotemark\footnotetext{\cite{Greggersen}}.

\begin{figure}[h]
 \centering
 \includegraphics[width=12cm]{"Linear und Kreislaufwirtschaft.png"}
 \caption{Vergleich der Prozessketten in Linear- und Kreislaufwirtschaft\protect\footnotemark}
 \label{fig:LinearKreislaufwirtschaft}
\end{figure}
\footnotetext{\cite{Gruenes_Kino}}

Im Folgenden wird nun die Umsetzung der Kreislaufwirtschaft in den folgenden Punkten betrachtet:

\begin{itemize}
  \item Produktdesign
  \item Produktionsprozesse
  \item Logistik
  \item Entsorgung
\end{itemize}
\newpage
\section{Produktdesign}
\subsection {Definition}
	Unter Produktdesign versteht man das Festlegen der Produkterscheinung in Hinblick auf Qualit\"at, Form, Verpackung und Markierung abh\"angig von der Produktart\protect\footnotemark\footnotetext{\cite{GablerWirtschaftslexikon}} im Rahmen der Produktpolitik. 
Früher war die Prim\"araufgabe des Produktdesigns gebunden an eine m\"oglichst optimale Funktionalit\"at der Produkte. 
Mit der Industrialisierung fand aufgrund der Markts\"attigung der \"Ubergang von der Produktions- zur Absatzorientierung statt.
Heute steht die absatzwirtschafliche Funktion des Produktdesign im Vordergrund um neue Märkte zu erschließen und Marktanteile auszubauen. 
Zudem rückt die kommunikative Wirkung der Produktpolitik in den Vordergrund mit dem Ergebnis des Erlebnis-Marketings. 
Firmen, wie Apple, versuchen in Konsumenten durch ihr Produktdesign einen gewissen Wunsch nach sozialer Anerkennung und Selbstverwirklichung auszul\"osen. 
Die Befriedigung dieser sozial-kulturellen Bed\"urfnisse stellt einen nicht zu vernachl\"assigbaren Wettbewerbsfaktor dar.\\

Im folgenden wird nun das Produktdesign hinsichtlich des Umweltgedanken n\"aher betrachtet.

\subsection {Einfluss des Produktdesigns auf den \"okologischen Fußabdruck}
	Das Produktdesign ist die kritischste Phase bei der Bewertung des Produktlebenszyklus(vgl. Abb. \ref{fig:Produktlebenszyklus}).
Die Entscheidungen, die w\"ahrend dieser Phase getroffen werden, legen die Materialien, Qualit\"at, Kosten, Prozesse, das Verpackungsmaterial, die Logistik, sowie den Entsorgungsprozess fest. 
Die getroffenen Entscheidungen haben deshalb einen enormen Einfluss auf den ökologischen Fußabdruck des Produktes. 
Beispielsweise, ob das Material hochwertig oder geringere Qualit\"at aufweist, oder biologisch abbaubare Materialien oder Plastikstoffe verwendet werden. 
Auch wird beispielsweise entschieden, wie das Produkt in der Degenerationsphase entsorgt und in welchem Grad das Produkt recycelnd werden kann. 
	\begin{figure}[h]
	 \centering
	 \includegraphics[width=10cm]{"Produktlebenszyklus.png"}
	 \caption{Die fünf Phasen des Produktlebenszyklus\protect\footnotemark}
	 \label{fig:Produktlebenszyklus}
	\end{figure}
	\footnotetext{\cite{Wikipedia}}

	All das macht das Produktdesign zum wichtigsten Ansatzpunkt um den ökologischen Fußabdrucks eines Produkts zu reduzieren. 
Dieser Ansatz kann durch umweltfreundliche Entscheidungen die Menge an Abfall, die Energiekosten für Hersteller und Logistiksystem, aber auch für den Endkonsumenten reduzieren.\\
Procter\&Gamble konnten durch diesen Ansatz, also der Betrachtung des Produkts und seiner Umweltfolgen bzw. Langzeitfolgen schon in der Designphase, den „Tide Coldwater“ entwickeln. 
Dieses Waschmittel kann auch mit kaltem Wasser verwendet werden und so dem Konsumenten ungefähr ¾ der Energiekosten eines gew\"ohnlichen Waschgangs sparen.
	
	\begin{figure}[h]
	 \begin{minipage}[b]{.4\linewidth} % [b] => Ausrichtung an \caption
	      \includegraphics[width=7cm]{Winglet.png}
	      \caption{Eine Designänderung ist der Winglet. Diese erhöhen die Steiggeschwindigkeit und reduzieren die Treibstoffkosten. Der Kraftstoffverbrauch kann so um drei bis f\"unf Prozent gesenkt werden\protect\footnotemark  Bild\protect\footnotemark}
	   \end{minipage}
	   \hspace{0.2\linewidth}% Abstand zwischen Bilder
	   \begin{minipage}[b]{.4\linewidth} % [b] => Ausrichtung an \caption
	      \includegraphics[width=7cm]{Kaffetabs.png}
	      \caption{Die La-Coppa-Kapseln funktionieren wie herk\"omliche Nespresso-Kapseln, kommen jedoch ohne Plastikverpackung asu und sind kompostierbar\protect\footnotemark}
	   \end{minipage}
	\end{figure}
	\footnotetext{\cite{Boeing}}
	\footnotetext{\cite{Airbus}}
	\footnotetext{\cite{Stern}}

	Andere erfolgreiche Designlösungen:
	\begin{itemize}
	  \item Das Boston Park Plaza Hotel konnte durch die Installation von Pumpspendern in ihren Badezimmern, die in Plastik verpackten Seifen und Shampoo Flaschen ($\approx$ 1 Million Plastikcontainer) einsparen
	  \item UPS konnte durch die Entwicklung eines wiederverwertbaren Expressumschlags die Menge an Herstellungsmaterial reduzieren. Diese Umschläge können vor dem Recycling 2mal benutzt werden
	  \item Coca-Cola reduzierte die Menge des ben\"otigten Plastiks für die Dasani Flaschen durch ein Re-Design. Durch die \"Anderung wurden die Flaschen 30\% leichter als bei Markteinführung
	\end{itemize}

\subsection {Alternative umweltfreundliche Designl\"osungen}
	Produkt Designer suchen auch nach alternativen Materialien zur Herstellung ihrer Produkte. 
Innovationen mit alternativen Materialien können teuer sein, machen aber umweltfreundlichere Autos, LKW und Flugzeuge durch verbesserte Lasten- und Treibstoff Effizienz möglich. 
Flugzeug- und Automobilhersteller beispielsweise suchen nach leichteren Materialien für ihre Produkte. 
Leichtere Materialien ermöglichen eine höhere Treibstoff Effizienz, weniger Abgase und reduzieren die Betriebskosten.\\
	
	Beispiele:
	\begin{itemize}
	  \item Mercedes baut verschiedene Karosserieaußenteile aus Bananenfasern. Diese beispielsweise T\"urblenden sind biologisch abbaubar und zudem leichter
	  \item Boeing benutzt Karbonfasern, Verbundstoffe und Titan-Graphit Schichtstoffe, um das Gesamtgewicht des 787 Dreamliners zu verringern und somit Treibstoff einzusparen
	  \item Celebrity Cruises baut seine Schiffe mit Energieeffizient als Kriterium und versucht ihre Kreuzfahrtschiffe m\"oglichst umweltfreundlich herzustellen und zu betreiben
	\end{itemize}
	
	\paragraph{Beispiel S1 Celebrity Cruises: „Save the Waves”\protect\footnotemark:}
	\begin{itemize}
	  \item Vermeidung, Recycling, Wiederverwertung - Reduzierung der Müllmenge und Einsatz von Recyling, wo auch immer m\"oglich
	  \item Müllvermeidung und Müllverarbeitung an Bord – Nichts wird ins Meer abgeleitet
	  \item Gesetzliche Vorgaben werden sogar noch strenger ausgelegt
	  \item Kontinuierliche Verbesserung - Wandel ist eine Konstante; Innovationen werden ausgenutzt und angestrebt
	\end{itemize}
	\footnotetext{\cite{CelebrityCruises}}
	
	\paragraph{Beispiel S2 MSC Cruises\protect\footnotemark:}
	\begin{itemize}
	  \item MSC Kreuzfahrtschiffe sind mit Verbrennungsanlagen, Mülltrennungsanlagen und Verarbeitungsanlagen ausgestattet. Giftstoffe und Gefahrstoffe werden am Hafen entsorgt
	  \item Reduzierung von Plastik an Bord. Wo möglich dies durch Alternativstoffe ausgetauscht. Z.B. Plastikstrohhalme durch Metallstrohhalme
	  \item Alternative Antriebssysteme, wie Brennstoffzellen oder Gas. Installation von Abgasreinigungssystemen (EGCS) um Abgase zu reinigen und den Kohlenstoffdioxidanteil zu verringern
	  \item Automatisiertes Abschalten von Klimaanlage und Licht in den einzelnen Kabinen bei Nichtbenutzung. Die Bordkarte ist der an/aus Schalter
	\end{itemize}
	\footnotetext{\cite{CruiseCritic}}
	
	Aber auch andere Unternehmen haben in den letzten Jahren den Umweltschutz als zentrales Ziel der Gesellschaft in ihre Geschäftsprozesse und Produkte mit aufgenommen. 
Produktdesigner müssen sich jedoch oft zwischen umweltfreundlichen und konventionellen Designalternativen entscheiden.
Aber auch andere Gesichtspunkte, wie Kaufverhalten und Preisstrukturen spielen einen wichtigen Faktor in Designentscheidungen.
Im folgenden wird anhand von Elektronikger\"aten die Geplante Obsoleszenz kurz dargestellt und warum sie ein wichtiger Gesichtspunkt des Produktdesigns ist.

\subsection {Geplante Obsoleszenz}
	\subsubsection {Definition}
	Obsoleszenz beschreibt den nat\"urlichen oder k\"unstlichen Alterungsprozess von Produkten, das dadurch veraltet oder unbrauchbar werden. 
Das Bedeutet ein Produkt kann die Bed\"urfnisbefriedigung nicht mehr sicherstellen.
Geplante Obsoleszenz liegt vor, wenn die Nutzungsdauer eines Produktes bewusst verkürzt oder niedrig gehalten wird. 
Dadurch soll der Produktabsatz vergrößert werden, da Kunden die Waren \"ofter kaufen m\"ussen und h\"aufiger das Produkt wechseln. 
Diese Marketingstrategie ist ein Hauptangriffspunkt des Verbraucherschutzes, wobei sich die Kritik hauptsächlich auf die psychologische Obsoleszenz fokussiert. 
Aber auch die im Volksmund vertretenen Sollbruchstellen sind wahrgenommene Resultate dieser Marketingstrategie und es stellt sich die Frage, warum heutzutage die Produkte immer noch geringer Lebensdauer haben. 
So wird oft gefragt, warum zum Beispiel die Weltraumsonde Voyager 40 Jahre im Weltraum funktionieren kann, eine normale Waschmaschine im Durchschnitt aber nur einen vierjährigen Lebenszeitraum hat, bevor die Maschine die Funktionsf\"ahigkeit verliert. 
Diese Vermutungen aufzukl\"ahren ist nicht einfach und auch verschiedene Studien, beispielsweise des Umweltbundesamtes\protect\footnotemark\footnotetext{\cite{Umweltbundesamtes}}, konnten die geplante Obsoleszenz nicht mit absoluter Sicherheit feststellen.
Die geplante Obsoleszenz kann in vier Arten klassifiziert werden:
	
	\paragraph {werkstofftechnische:}
	Werkstofftechnische Obsoleszenz liegt vor, wenn einige Kleinteile des Produkts weniger leistungsfähig als andere sind und schneller ausfallen. Beispiele sind der Smartphoneakku oder das Display.\protect\footnotemark\footnotetext{\cite{WERTGARANTIE}; 16.12.2020}. Die Produktalterung kann sich in einem vollständigen Funktionsverlust oder in einer Leistungsminderung widerspiegeln.
	\paragraph {technologische/technische:}
	Unter technischer Obsoleszenz wird versteht man das Veralten eines Produkts durch ein neues Produkt, was seine Funktion besser erfüllt. Ursachen sind die sich rasch ändernden technischen Gegebenheiten und funktionalen M\"oglichkeiten.
Zum Beispiel kann Gerät eine neue Softwareversion nicht mehr ausführen, weil das Gerät die Mindestvoraussetzungen nicht mehr erfüllen kann.
	\paragraph {psychologische:}
	Die Psychologische Obsoleszenz beschreibt ein verbreitetes Marketinginstrument. 
Dabei wird, vor allem in Werbung, für verbesserte Technik geworben oder beispielsweise neue Modetrends gesetzt. 
Ein Beispiel hierfür ist Apple, das trotz teilweise nur geringen technischen Verbesserungen hauptsächlich auf optische Produktverbesserungen setzt. 
So unterscheiden sich verschiedene I-Phones teilweise technisch nur marginal voneinander. Auch wird diese sehr häufig von der Modeindustrie benutzt. 
So werden ständig neu Kollektionen als neue Trends angepriesen um Käufer anzuwerben und teilweise das gleiche T-Shirt oder einen Pullover in einer anderen Farbe zu verkaufen. 
Diese psychologische Einwirken durch Werbung soll dabei einen Zusatznutzen vermitteln und das alte Kleidungsst\"uck abwerten.
	\paragraph {\"okonomische:}
	Bei der \"okonomischen Obsoleszenz wird die Reparatur oder Instandsetzung eines Produkts aus Kostengr\"unden verhindert. 
Dieser Aspekt der Obsoleszenz ist ein typisches Symptom der Wegwerfgesellschaft, wo teilweise noch reparierbare, vor allem elektrische Ger\"ate aus Kostengr\"unden nicht repariert werden.\\
	
	\subsubsection {Unterscheidung von Nat\"urliche Obsoleszenz und Nutzungsobsoleszenz}
	Neben der geplanten Obsoleszenz gibt es auch die nat\"urliche Alterung, zum Beispiel aufgrund von material- und nutzungsbedingten Qualitätsverlusten. 
Sie h\"angt von der Materialbest\"andigkeit ab, also wie stark sich ein Produkt durch den normalen Gebrauch abnutzt. 
Beispielsweise f\"uhrt der Wiederaufladevorgang einer Handybatterie nachweislich zu geringf\"ugigen Kapazit\"atsverlusten\protect\footnotemark\footnotetext{\cite{Prenner}}. 
Das ist in diesem Fall keine geplante Obsoleszenz, sondern der Funktionsweise von Akkus geschuldet.\\
Nutzungsobsoleszenz beschreibt, das Konsumenten ein vollständiges und uneingeschr\"ankt nutzbares Produkt vorzeitig nicht mehr verwenden.\protect\footnotemark\footnotetext{\cite{Gutberlet}}. 
Zum Beispiel, dass man sich als Ersatzprodukt f\"ur einen Fernseher, einen gr\"oßeren kauft, obwohl er noch funktioniert.
	\subsubsection {Folgen von Obsoleszenz}
	Die Folgen von geplanter Obsoleszenz sind vielfältig. So sorgen die immer kürzeren Gebrauchtzeiten für einen rapiden Anstieg von Elektroschrott, der teilweise nicht, wie gesetzlich verboten, illegal in Müllkippen im Ausland entsorgt wird. 
Die Folgen für Mensch und Natur sind fatal. Teilweise gelangen durch die in Elektronik verbauten Elemente giftige Stoffe, wie Quecksilber und Kadmium, in den Boden und in das Grundwasser. 
Auch gelangen durch diese illegale Entsorgung Plastikstoffe in die Umwelt.
Afrika und \"armere Staaten in Asien sind davon besonders betroffen.\protect\footnotemark\footnotetext{\cite{Spiegel}} \\
	Dieser Elektroschrott wird dadurch ebenfalls nicht dem Recyclingsystem zugeführt. 
Dadurch gehen wertvolle, knappe Ressourcen, wie Edelmetalle und seltene Erden verloren, die dann wieder teurer aus dem Ausland importiert werden müssen. 

	\subsubsection {Strategien gegen Obsoleszenz}
	Nun stellt sich die Frage, wie man diese negativen Entwicklungen verbessern bzw. verhindern kann. 
Es bieten sich die folgenden M\"oglichkeiten an:
	\paragraph {Reparieren}
	Die wichtigsten Maßnahmen sind die Sicherstellung der Reparaturm\"oglichkeiten von Produkten, zum Beispiel durch ein Modularsystem,  und durch die Reduzierung der Reparaturkosten. 
Zus\"atzlich muss es die M\"oglichkeit geben, diese Ersatzteile auch zu erwerben. 
Den nur wenn eine Reparatur in einem nachvollziehbaren Verh\"altnis zum Wiederkaufspresi steht, sind Konsumenten auch gewillt das Produkt reparieren zu lassen oder das Produkt selbst mit Ersatzteilen zu reparieren. 
	\paragraph {Bewusstes Kaufverhalten}
	Nicht jedes Produkt gleicher Art ist gleich. Die Haltbarkeit und die Modularf\"ahigkeit der Produkte unterscheiden sich stark, daneben unterscheiden sich auch die M\"oglichkeiten der Entsorgung bzw. die Recyclingm\"oglichkeiten. 
Bewusstes Kaufen setzt auch Eigenverantwortung voraus, zum Beispiel kann man sich \"uber verschiedene Ratgeber, wie Gr\"une Elektronik von Greenpeece\protect\footnotemark\footnotetext{\cite{Greenpeace}; 16.12.2020}, selbstst\"andig informieren.
Man kann verschiedene Produkte vergleichen und sich dann f\"ur ein Produkt bewusst entscheiden.
	\begin{figure}[h]
	 \centering
	 \includegraphics[width=10cm]{"Guide to Greener Elektronics.png"}
	 \caption{Greenpeace-Ranking zu grüner Elektronik\protect\footnotemark}
	 \label{fig:GreenerElektronics}
	\end{figure}
	\footnotetext{\cite{GreenpeaceRanking};17.12.2020}	
	\paragraph {Gebrauchtwaren}
	Eine weitere M\"oglichkeit ist, das Kaufen, Verkaufen, Verleihen oder auch Spenden von Produkten, die noch funktionsf\"ahig sind oder nur geringe M\"angel aufweisen. 
Dieser Handel von Gebrauchtwaren spart Ressourcen und man spart Geld. 
Zur Anwendung kann man verschiedene Tauschb\"orsen doer auch die Kleinanzeigen in der Zeitung verwenden.
	\paragraph {Entsorgen}
	Die R\"ucknahme von Elektronikartikeln und umweltvertr\"agliche Entsorgung regelt das sogenannte Elektronikgesetz (Gesetz \"uber das Inverkehrbringen, die R\"ucknahme und die umweltvertr\"agliche Entsorgung von Elektro- und Elektronikger\"aten). 
Gem\"aß Gesetz sind die Verk\"aufer von Elektronikartikeln verpflichtet, diese wieder nach Ausmusterung zur\"uckzunehmen. 
Auch kann man sie kostenlos an jedem Recyclinghof oder Werkstoffhof abgeben.
	\subsubsection {Fazit}
	Die Bewertung geplanter Obsoleszenz ist schwierig. 
Produzenten m\"ochten ihre Produkte m\"oglichst gewinnbringend verkaufen und kosteng\"unstig herstellen.
Die Kunden dagegen billige, aber hochwertige, gut gestaltete Produkte. 
Es herrscht eine Art Status Quo zwischen Kunden- und Herstelleransichten und Erwartungen.
Folglich kann die geplante Obsoleszenz oftmals als notwendiges Übel angesehen werden um Kunden und Herstelleransichten miteinander in Einklang zu bringen. 
Sie kann auch nie ganz ausger\"aumt werden, weil teilweise die Einordnung, ob geplante Obsoleszenz vorliegt oder nur eine nat\"urliche Obsoleszenz vorliegt nicht eindeutig gekl\"art werden kann.

\subsection {Beeinflussung des Produktdesigns durch das Kaufverhalten}
	Ein weiterer Faktor in der Beurteilung von Entscheidungen in der Designphase ist, das Kaufverhalten und die Erstreaktion des potentiellen Kunden auf das Produkt abzusch\"atzen und einzuordnen.
Traditionelle Kriterien, wie gutes Design, Haltbarkeit und auch Funktionalit\"at sind, trotz dem in den letzen st\"arkeren Umweltgedanken, immer noch die kaufentscheidenden Kriterien. 
Der Umweltschutz ist ein Sekund\"arfaktor und wird eher als Zusatznutzen wahrgenommen.
Um sie in den Vordergrund der Kaufentscheidung zu r\"ucken muss die Kommunikation diesbez\"uglich klarer darauf fokusiert werden, z.B. durch Siegel oder QR-Codes.
Laut einer Studie des Umweltbundesamtes sind die Entsorgung und die Produktion von Produkten für die meisten K\"aufer bei Nicht-Verbrauchsg\"utern die wichtigsten Kriterien in der Nachhaltigkeitsbilanz. 
Die Nutzungsdauer stellte sich durch die Untersuchung als wichtigster Faktor in nachhaltigem Konsum dar.\protect\footnotemark\footnotetext{\cite{Umweltbundesamt2019}; 16.12.2020}\\ 
Weitere Faktoren sind beispielsweise Schadstofffreiheit, nachhaltige Rohstoffe und Recyclingm\"oglichkeiten. 
Das Potenzial umweltfreundlich entworfener Produkte ist also vorhanden, es muss nur klarer kommuniziert und umgesetzt werden.
Zum Beispiel ist der Trend bei Laptops zu immer d\"unneren, leichteren Ger\"ate, die dadurch oft verklebt und nicht modular aufgebaut sind. 
Dadurch ist die eigenh\"andige Raperatur fast unm\"oglich, nur mit Fachwissen druchf\"uhrbar und ohne Entfall der Gew\"ahrleistung durch das \"Offnen nicht m\"oglich.
Das etablieren eines Modularsystems und die M\"oglichkeit an g\"unstige Ersatzteile zu kommen, k\"onnte diese Produkte f\"ur den Kunden attraktiv machen und ihn zum Kauf anregen.
\subsection {Beispiel - Smartphones}
	Smartphones sind heute Konsumg\"uter.
Ihre Nutzungszeit ist dabei an ihre kurze Akkulebensdauer und die immer h\"oher werdenden Softwareanforderungen gebunden.
Daduch k\"onnen sie nur wenige Jahre genutzt werden.
Neue Applikationen ben\"otigen immer h\"ohere Ger\"ateleistung und das Handy wird immer mehr zum techniklastigen Multimediager\"at.
Neue Trends und Designans\"atze verk\"urzen dieses Kaufverhalten durch psychische Obsoleszenz zus\"atzlich, in dem sie in dem K\"aufer durch Werbung einen Zusatznutzen, wie Selbstbewusstsein und Technikaffinit\"at, in Neuger\"aten suggerieren.
Das neuste Smartphone wurde zum Statussymbol, aber damit auch zum Wegwerfprodukt, das schnell ausgetauscht werden kann, wenn es diesen verliert.
Zwar werden die Smartphones immer leistungss\"arker, oftmals steht dabei das Neudesign im Vordergrund.
Um die Designanforderungen, also immer d\"unner, leichter und lesitungsf\"ahiger, umzusetzen, sowie optisch ansprechend zu machen, sind die Ger\"ate immer kompakter und komplexer in ihrem Aufbau geworden. 
Die Modulartechnik fr\"uherer Modelle wurde dazu aufgegeben und Hersteller neigen immer mehr dazu, die Ger\"ate zu Verkleben und verschiedene Einheiten, wie Akku und Sensoren, zu verbinden um immer noch d\"unnere, in der Produktion kosteng\"unstigere Ger\"ate einfach herzustellen.
So waren \"altere Samsung-Smartphones immer mit einem austauschbaren Akku ausgestattet, wie zum Beispiel das Galaxy S5. Ging die Akkuleistung zur\"uck, konnte man kosteng\"unstig einen neuen Akku extern kaufen. 
Bei einem Neupreis eines Samsung Galaxy S5(Neupreis 200\euro{}) stand das mit einem Kauf eines Originalakkus zu Preis von 10\euro{} in einem gesunden Verh\"altnis.
Vergleicht man das mit dem Akkutausch eines Galaxy S6, so wird die Tendenz zu \"okonomischer Obsoleszenz deutlich. Mit einem Neupreis von 200\euro{} ist es im Schnitt genauso teuer wie das Vorg\"angermodell. 
Der Akkutausch kostet im Gegensatz dazu 50\euro{}, ein Viertel des Neupreises. Im Gegensatz dazu das S5 mit nth{20} des Neupreises.
Geschuldet ist dies dem nicht vorhandenen Modularsystem, der Akku ist fest in das Ger\"at intergriert, und dem komplexeren Aufbau des Galaxy S6. 
Zudem kann der Akku nicht mehr einfach durch den Laien ausgetauscht werden und bedarf Fachwissen.
Die Konsequenz ist, das Kunden bei schwacher Akuuleistung, sich ein neues Smartphone-Modell kaufen und das alte, eigentlich noch funktionierende, Ger\"at entsorgen und nicht nur den Akku austauschen.
\subsection {Fazit}
	Die Betrachtung des Produktdesigns in Bezug auf die Reduzierung des \"okologischen Fußabdrucks ist vielseitig und vielschichtig. 
An ein Produkt sind gewisse kundenspezifische Erwartungen gebunden und die umweltfreundliche Umzusetzung ist nicht einfach und teilweise auch nicht \"okonomisch umsetzbar. 
Hochwertige, umweltfreundliche Produkte sind in der Regel teurer als herk\"omliche Produkte und haben dadurch einen Wettbewerbsnachteil und sind durch optische Einschr\"ankungen schwerer zu vermarkten. 
Sie m\"ussen designgebundene Abstriche machen, die von Kunden negativ interpretiert werden k\"onnten, um dennoch die erwarteten Funktionalit\"aten umzusetzen.\\
Neben den Herstellern sind aber auch die Konsumenten Schuld an dieser Entwicklung. 
Mit ihrem unbewussten Kaufverhalten neigen sie oftmals zur billigen Konkurrenz und greifen zum neusten Trendgegenstand, ohne sich \"uber die Umweltfreundlichkeit Gedanken zu machen oder sich \"uber umweltfreundlicheren Alternativen zu informieren.
Sie lassen sich zu einfach durch Werbung zum Kauf animieren und informieren sich zu wenig \"uber die Konsumgegenst\"ande sie sie kaufen. Designtrends haben immer noch Vorrang vor umweltfreundlichen Produktdesign.
Zwar ist in den letzten Jahren die Bereitschaft zum umweltbewussten Kauf gestiegen und auch der Umweltschutz hat heute an Stellenwert bei Kaufentscheidungen gewonnen.
Um jedoch den Umweltaspekt als kaufentscheidendes Kriterium zu etablieren, ist es noch ein langer Weg. 

\newpage

\printbibliography

\end{document}